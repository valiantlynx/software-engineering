\documentclass[a4paper, 12pt]{article}
\usepackage[margin=1in]{geometry}
\usepackage{setspace}

% Begin Document
\begin{document}
\noindent
\textbf{Brooks - No Silver Bullet}

\vspace{0.5cm}

\textbf{Gormery K. Wanjiru}

\vspace{0.5cm}

In \textit{"No Silver Bullet"}, Brooks argues that software development’s core challenges are inherent and cannot be easily eliminated by any single technology or method. He distinguishes between \textit{essential} difficulties (inherent complexity, changeability, conformity, invisibility) and \textit{accidental} difficulties (technical limitations or inefficiencies). Brooks suggests that while technology has helped reduce accidental difficulties, essential complexities remain stubborn and resistant to radical improvements.

Brooks sees software as uniquely complex because it involves abstract constructs that are deeply interdependent, making it hard to simplify. For example, he points out that while hardware has seen exponential improvements, software lacks similar gains due to its inherent complexity and reliance on human creativity and insight. Brooks proposes solutions like prototyping, incremental development, and nurturing “great designers,” emphasizing that managing complexity requires more than just tools or methods.

\vspace{0.5cm}

Brooks’s analysis remains relevant, especially his view on essential difficulties. However, he underestimated some advances, like Object-Oriented Programming (OOP), which has proven crucial in tackling complexity by promoting modularity and reusability. It simplifies complexity and allows for easier collaboration. Additionally, he doesn’t fully account for open-source software's rise, which has transformed collaboration and accelerated innovation, addressing some of the very issues he highlighted.

While Brooks’s focus on “great designers” is valid, modern practices, like Agile and DevOps, show that effective collaboration can be just as impactful. Although no single technology has provided the “silver bullet” Brooks warned against, combining these methodologies with skilled teams has proven effective in managing the complexities of software engineering.

Also today there are software technologies advancements that have started to help with reducing essential difficulties. Technologies like AI and Devops especially in modern time are most noteworthy. Different from what he meant there have been no "radical improvements" in both these

\end{document}
