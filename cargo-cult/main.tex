\documentclass[a4paper, 12pt]{article}
\usepackage[margin=1in]{geometry}
\usepackage{setspace}

% Begin Document
\begin{document}
\noindent
\textbf{McConnell's "Cargo Cult Software Engineering"}

\vspace{0.5cm}

\textbf{Gormery K. Wanjiru}

\vspace{0.5cm}

McConnell’s \textit{"Cargo Cult Software Engineering"} compares cargo cults, which imitate without understanding, to software teams that copy development practices without truly understanding their true value. He puts forward two styles of development: process-oriented and commitment-oriented. Process-oriented teams focus on careful planning, process improvement, and efficient use of resources, while commitment-oriented teams rely on motivated individuals who work long hours to complete projects.\\

The problem, McConnell says, is when organizations imitate these methods without understanding them. Some companies become "bureaucratic imposters," thinking that just generating documentation or holding meetings will lead to success. Others become "sweatshops," believing that forcing employees to work long hours without proper motivation will yield good results. McConnell calls both of these "cargo cult" organizations because they mimic the appearance of successful companies without understanding the key principles that make them work.\\

McConnell argues that the real debate isn't about whether process or commitment is better, but about understanding what makes a project succeed. Companies should focus on increasing the competence of their teams, not just following rigid processes or overworking employees.

\vspace{0.5cm}

\noindent
\textbf{Critique}

\vspace{0.5cm}

McConnell presents a real problem. I fully agree with his point, especially when he says people think that if you follow a process-driven approach, you can't also have commitment, and vice versa. He explains it in a way that makes sense. From my own experience, it’s hard to explain this to someone who misunderstands. A company can choose whether they are process or commitment-driven, but many don’t. I would suggest using practices but changing as the team or company evolves. McConnell hints at this, and it’s what we see in modern companies that are successful. What’s common in all of this is the need for an understanding of software engineering principles.

\end{document}
